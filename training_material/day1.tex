% Options for packages loaded elsewhere
\PassOptionsToPackage{unicode}{hyperref}
\PassOptionsToPackage{hyphens}{url}
%
\documentclass[
  ignorenonframetext,
]{beamer}
\usepackage{pgfpages}
\setbeamertemplate{caption}[numbered]
\setbeamertemplate{caption label separator}{: }
\setbeamercolor{caption name}{fg=normal text.fg}
\beamertemplatenavigationsymbolsempty
% Prevent slide breaks in the middle of a paragraph
\widowpenalties 1 10000
\raggedbottom
\setbeamertemplate{part page}{
  \centering
  \begin{beamercolorbox}[sep=16pt,center]{part title}
    \usebeamerfont{part title}\insertpart\par
  \end{beamercolorbox}
}
\setbeamertemplate{section page}{
  \centering
  \begin{beamercolorbox}[sep=12pt,center]{part title}
    \usebeamerfont{section title}\insertsection\par
  \end{beamercolorbox}
}
\setbeamertemplate{subsection page}{
  \centering
  \begin{beamercolorbox}[sep=8pt,center]{part title}
    \usebeamerfont{subsection title}\insertsubsection\par
  \end{beamercolorbox}
}
\AtBeginPart{
  \frame{\partpage}
}
\AtBeginSection{
  \ifbibliography
  \else
    \frame{\sectionpage}
  \fi
}
\AtBeginSubsection{
  \frame{\subsectionpage}
}
\usepackage{amsmath,amssymb}
\usepackage{iftex}
\ifPDFTeX
  \usepackage[T1]{fontenc}
  \usepackage[utf8]{inputenc}
  \usepackage{textcomp} % provide euro and other symbols
\else % if luatex or xetex
  \usepackage{unicode-math} % this also loads fontspec
  \defaultfontfeatures{Scale=MatchLowercase}
  \defaultfontfeatures[\rmfamily]{Ligatures=TeX,Scale=1}
\fi
\usepackage{lmodern}
\ifPDFTeX\else
  % xetex/luatex font selection
\fi
% Use upquote if available, for straight quotes in verbatim environments
\IfFileExists{upquote.sty}{\usepackage{upquote}}{}
\IfFileExists{microtype.sty}{% use microtype if available
  \usepackage[]{microtype}
  \UseMicrotypeSet[protrusion]{basicmath} % disable protrusion for tt fonts
}{}
\makeatletter
\@ifundefined{KOMAClassName}{% if non-KOMA class
  \IfFileExists{parskip.sty}{%
    \usepackage{parskip}
  }{% else
    \setlength{\parindent}{0pt}
    \setlength{\parskip}{6pt plus 2pt minus 1pt}}
}{% if KOMA class
  \KOMAoptions{parskip=half}}
\makeatother
\usepackage{xcolor}
\newif\ifbibliography
\usepackage{color}
\usepackage{fancyvrb}
\newcommand{\VerbBar}{|}
\newcommand{\VERB}{\Verb[commandchars=\\\{\}]}
\DefineVerbatimEnvironment{Highlighting}{Verbatim}{commandchars=\\\{\}}
% Add ',fontsize=\small' for more characters per line
\usepackage{framed}
\definecolor{shadecolor}{RGB}{248,248,248}
\newenvironment{Shaded}{\begin{snugshade}}{\end{snugshade}}
\newcommand{\AlertTok}[1]{\textcolor[rgb]{0.94,0.16,0.16}{#1}}
\newcommand{\AnnotationTok}[1]{\textcolor[rgb]{0.56,0.35,0.01}{\textbf{\textit{#1}}}}
\newcommand{\AttributeTok}[1]{\textcolor[rgb]{0.13,0.29,0.53}{#1}}
\newcommand{\BaseNTok}[1]{\textcolor[rgb]{0.00,0.00,0.81}{#1}}
\newcommand{\BuiltInTok}[1]{#1}
\newcommand{\CharTok}[1]{\textcolor[rgb]{0.31,0.60,0.02}{#1}}
\newcommand{\CommentTok}[1]{\textcolor[rgb]{0.56,0.35,0.01}{\textit{#1}}}
\newcommand{\CommentVarTok}[1]{\textcolor[rgb]{0.56,0.35,0.01}{\textbf{\textit{#1}}}}
\newcommand{\ConstantTok}[1]{\textcolor[rgb]{0.56,0.35,0.01}{#1}}
\newcommand{\ControlFlowTok}[1]{\textcolor[rgb]{0.13,0.29,0.53}{\textbf{#1}}}
\newcommand{\DataTypeTok}[1]{\textcolor[rgb]{0.13,0.29,0.53}{#1}}
\newcommand{\DecValTok}[1]{\textcolor[rgb]{0.00,0.00,0.81}{#1}}
\newcommand{\DocumentationTok}[1]{\textcolor[rgb]{0.56,0.35,0.01}{\textbf{\textit{#1}}}}
\newcommand{\ErrorTok}[1]{\textcolor[rgb]{0.64,0.00,0.00}{\textbf{#1}}}
\newcommand{\ExtensionTok}[1]{#1}
\newcommand{\FloatTok}[1]{\textcolor[rgb]{0.00,0.00,0.81}{#1}}
\newcommand{\FunctionTok}[1]{\textcolor[rgb]{0.13,0.29,0.53}{\textbf{#1}}}
\newcommand{\ImportTok}[1]{#1}
\newcommand{\InformationTok}[1]{\textcolor[rgb]{0.56,0.35,0.01}{\textbf{\textit{#1}}}}
\newcommand{\KeywordTok}[1]{\textcolor[rgb]{0.13,0.29,0.53}{\textbf{#1}}}
\newcommand{\NormalTok}[1]{#1}
\newcommand{\OperatorTok}[1]{\textcolor[rgb]{0.81,0.36,0.00}{\textbf{#1}}}
\newcommand{\OtherTok}[1]{\textcolor[rgb]{0.56,0.35,0.01}{#1}}
\newcommand{\PreprocessorTok}[1]{\textcolor[rgb]{0.56,0.35,0.01}{\textit{#1}}}
\newcommand{\RegionMarkerTok}[1]{#1}
\newcommand{\SpecialCharTok}[1]{\textcolor[rgb]{0.81,0.36,0.00}{\textbf{#1}}}
\newcommand{\SpecialStringTok}[1]{\textcolor[rgb]{0.31,0.60,0.02}{#1}}
\newcommand{\StringTok}[1]{\textcolor[rgb]{0.31,0.60,0.02}{#1}}
\newcommand{\VariableTok}[1]{\textcolor[rgb]{0.00,0.00,0.00}{#1}}
\newcommand{\VerbatimStringTok}[1]{\textcolor[rgb]{0.31,0.60,0.02}{#1}}
\newcommand{\WarningTok}[1]{\textcolor[rgb]{0.56,0.35,0.01}{\textbf{\textit{#1}}}}
\usepackage{longtable,booktabs,array}
\usepackage{calc} % for calculating minipage widths
\usepackage{caption}
% Make caption package work with longtable
\makeatletter
\def\fnum@table{\tablename~\thetable}
\makeatother
\usepackage{graphicx}
\makeatletter
\def\maxwidth{\ifdim\Gin@nat@width>\linewidth\linewidth\else\Gin@nat@width\fi}
\def\maxheight{\ifdim\Gin@nat@height>\textheight\textheight\else\Gin@nat@height\fi}
\makeatother
% Scale images if necessary, so that they will not overflow the page
% margins by default, and it is still possible to overwrite the defaults
% using explicit options in \includegraphics[width, height, ...]{}
\setkeys{Gin}{width=\maxwidth,height=\maxheight,keepaspectratio}
% Set default figure placement to htbp
\makeatletter
\def\fps@figure{htbp}
\makeatother
\setlength{\emergencystretch}{3em} % prevent overfull lines
\providecommand{\tightlist}{%
  \setlength{\itemsep}{0pt}\setlength{\parskip}{0pt}}
\setcounter{secnumdepth}{-\maxdimen} % remove section numbering
\ifLuaTeX
  \usepackage{selnolig}  % disable illegal ligatures
\fi
\IfFileExists{bookmark.sty}{\usepackage{bookmark}}{\usepackage{hyperref}}
\IfFileExists{xurl.sty}{\usepackage{xurl}}{} % add URL line breaks if available
\urlstyle{same}
\hypersetup{
  pdftitle={Introduction to Bayesian Disease Measurement for Health Scientists - Day 1},
  pdfauthor={Eleftherios Meletis; Polychronis Kostoulas},
  hidelinks,
  pdfcreator={LaTeX via pandoc}}

\title{Introduction to Bayesian Disease Measurement for Health
Scientists - Day 1}
\subtitle{CA18208 HARMONY Larissa Training School 2024 -
\url{https://harmony-net.eu/}}
\author{Eleftherios Meletis \and Polychronis Kostoulas}
\date{2024-04-22}

\begin{document}
\frame{\titlepage}

\begin{frame}{Bayesian Statistics}
\protect\hypertarget{bayesian-statistics}{}
\#In this session we'll see how we can estimate a probability of
interest but in a Bayesian framework, i.e.~using Bayes theorem.
\end{frame}

\begin{frame}{Bayes' theorem}
\protect\hypertarget{bayes-theorem}{}
\end{frame}

\begin{frame}{P(A\textbar B) = P(B\textbar A)*P(A)/P(B)}
\protect\hypertarget{pab-pbapapb}{}
\end{frame}

\begin{frame}{Components}
\protect\hypertarget{components}{}
\end{frame}

\begin{frame}{* P(A\textbar B): Prob of event A occurring given that B
is true - Posterior probability}
\protect\hypertarget{pab-prob-of-event-a-occurring-given-that-b-is-true---posterior-probability}{}
\end{frame}

\begin{frame}{* P(B\textbar A): Prob of event B occurring given that A
is true - Likelihood \textasciitilde{} function of A}
\protect\hypertarget{pba-prob-of-event-b-occurring-given-that-a-is-true---likelihood-function-of-a}{}
\end{frame}

\begin{frame}{* P(A): Prob of event A occurring - Prior probability}
\protect\hypertarget{pa-prob-of-event-a-occurring---prior-probability}{}
\end{frame}

\begin{frame}{* P(B): Prob of event B occurring - Marginal probability
\textasciitilde{} sum over all possible values of A}
\protect\hypertarget{pb-prob-of-event-b-occurring---marginal-probability-sum-over-all-possible-values-of-a}{}
\end{frame}

\begin{frame}{What we usually see/use}
\protect\hypertarget{what-we-usually-seeuse}{}
\end{frame}

\begin{frame}{theta: parameter of interest \textbar{} y: observed
data\}}
\protect\hypertarget{theta-parameter-of-interest-y-observed-data}{}
\end{frame}

\begin{frame}{P(theta\textbar y) = P(y\textbar{}\theta) * P(theta)/P(y)}
\protect\hypertarget{pthetay-py-pthetapy}{}
\end{frame}

\begin{frame}{Where:}
\protect\hypertarget{where}{}
\end{frame}

\begin{frame}{* P(theta): Prior probability of parameter(s) of
interest;}
\protect\hypertarget{ptheta-prior-probability-of-parameters-of-interest}{}
\end{frame}

\begin{frame}{* P(y\textbar theta): Likelihood of the data given the
parameters value(s)}
\protect\hypertarget{pytheta-likelihood-of-the-data-given-the-parameters-values}{}
\end{frame}

\begin{frame}{* P(theta\textbar y): Posterior probability of
parameter(s) of interest given the data and the prior}
\protect\hypertarget{pthetay-posterior-probability-of-parameters-of-interest-given-the-data-and-the-prior}{}
\begin{block}{Hui-Walter models}
\protect\hypertarget{hui-walter-models}{}
\begin{itemize}
\item
  A particular model formulation that was originally designed for
  evaluating diagnostic tests in the absence of a gold standard
\item
  Not originally/necessarily Bayesian - implemented using Maximum
  Likelihood
\item
  But evaluating an imperfect test against another imperfect test is a
  bit like pulling a rabbit out of a hat

  \begin{itemize}
  \tightlist
  \item
    If we don't know the true disease status, how can we estimate
    sensitivity or specificity for either test?
  \end{itemize}
\end{itemize}
\end{block}
\end{frame}

\begin{frame}
\begin{block}{\emph{Se} - \emph{Sp} estimation - Recap}
\protect\hypertarget{se---sp-estimation---recap}{}
When the true infectious status is known \emph{Se}-\emph{Sp} of test 1
can be estimated:

\includegraphics{figs/intro2x2.pdf}
\end{block}
\end{frame}

\begin{frame}
\begin{block}{Hui-Walter models (I)}
\protect\hypertarget{hui-walter-models-i}{}
\begin{itemize}
\item
  A particular model formulation that was originally designed for
  evaluating diagnostic tests in the absence of a gold standard
\item
  Also known as the two\_test - two\_population setting/paradigm
\end{itemize}

\includegraphics{figs/hui.walter.pdf}
\end{block}
\end{frame}

\begin{frame}
\begin{block}{Hui-Walter models (II)}
\protect\hypertarget{hui-walter-models-ii}{}
In the Hui-Walter paradigm a condition that \emph{connects} \(S\) the
number of populations (P) and \(R\) the number of tests (T) is desribed:

\[ S \geq \frac {R}{(2\textsuperscript{R-1} - 1)}\]

If this condition is fulfilled, any combination of \(S\) and \(R\) may
allow to estimate \(Se\) and \(Sp\), e.g.~(2T, 2P), (3T,1P), (4T,1P),
\ldots{}

\begin{itemize}
\tightlist
\item
  This condition describes \textbf{Model Identifiability} and is a
  necessary but not sufficient condition - as we'll see later.
\end{itemize}
\end{block}
\end{frame}

\begin{frame}[fragile]
\begin{block}{Hui-Walter models}
\protect\hypertarget{hui-walter-models-1}{}
\begin{itemize}
\tightlist
\item
  For demonstration purposes we will start with the two\_test -
  one\_population setting (\texttt{hw\_definition}) and continue adding
  both populations (\(S\)) and tests (\(R\)) along the way.
\end{itemize}

\begin{block}{Two\_test - One\_population setting}
\protect\hypertarget{two_test---one_population-setting}{}
\begin{itemize}
\item
  The data are summarized in a two\_x\_two table (2\^{}2 cells)
\item
  or in a vector that contains all possible test results combinations
\end{itemize}
\end{block}
\end{block}
\end{frame}

\begin{frame}[fragile]
\begin{block}{Model Specification (`hw\_definition')}
\protect\hypertarget{model-specification-hw_definition}{}
\scriptsize

\begin{verbatim}
model{
  Cross_Classified_Data ~ dmulti(prob, N)
  
  # Test1+ Test2+
    prob[1] <- (prev * ((se[1])*(se[2]))) + ((1-prev) * ((1-sp[1])*(1-sp[2])))
  
  # Test1+ Test2-
    prob[2] <- (prev * ((se[1])*(1-se[2]))) + ((1-prev) * ((1-sp[1])*(sp[2])))

  # Test1- Test2+
    prob[3] <- (prev * ((1-se[1])*(se[2]))) + ((1-prev) * ((sp[1])*(1-sp[2])))

  # Test1- Test2-
    prob[4] <- (prev * ((1-se[1])*(1-se[2]))) + ((1-prev) * ((sp[1])*(sp[2])))

  prev ~ dbeta(1, 1)
  se[1] ~ dbeta(1, 1)
  sp[1] ~ dbeta(1, 1)
  se[2] ~ dbeta(1, 1)
  sp[2] ~ dbeta(1, 1)

  #data# Cross_Classified_Data, N
  #monitor# prev, prob, se, sp
  #inits# prev, se, sp
}
\end{verbatim}

\normalsize
\end{block}
\end{frame}

\begin{frame}[fragile]
\scriptsize

\begin{Shaded}
\begin{Highlighting}[]
\NormalTok{twoXtwo }\OtherTok{\textless{}{-}} \FunctionTok{matrix}\NormalTok{(}\FunctionTok{c}\NormalTok{(}\DecValTok{36}\NormalTok{, }\DecValTok{4}\NormalTok{, }\DecValTok{12}\NormalTok{, }\DecValTok{48}\NormalTok{), }\AttributeTok{ncol=}\DecValTok{2}\NormalTok{, }\AttributeTok{nrow=}\DecValTok{2}\NormalTok{)}
\NormalTok{twoXtwo}
\DocumentationTok{\#\#      [,1] [,2]}
\DocumentationTok{\#\# [1,]   36   12}
\DocumentationTok{\#\# [2,]    4   48}
\end{Highlighting}
\end{Shaded}

\normalsize

\scriptsize

\begin{Shaded}
\begin{Highlighting}[]
\FunctionTok{library}\NormalTok{(}\StringTok{\textquotesingle{}runjags\textquotesingle{}}\NormalTok{)}

\NormalTok{Cross\_Classified\_Data }\OtherTok{\textless{}{-}} \FunctionTok{as.numeric}\NormalTok{(twoXtwo)}
\NormalTok{N }\OtherTok{\textless{}{-}} \FunctionTok{sum}\NormalTok{(Cross\_Classified\_Data)}

\NormalTok{prev }\OtherTok{\textless{}{-}} \FunctionTok{list}\NormalTok{(}\AttributeTok{chain1=}\FloatTok{0.05}\NormalTok{, }\AttributeTok{chain2=}\FloatTok{0.95}\NormalTok{)}
\NormalTok{se }\OtherTok{\textless{}{-}} \FunctionTok{list}\NormalTok{(}\AttributeTok{chain1=}\FunctionTok{c}\NormalTok{(}\FloatTok{0.01}\NormalTok{,}\FloatTok{0.99}\NormalTok{), }\AttributeTok{chain2=}\FunctionTok{c}\NormalTok{(}\FloatTok{0.99}\NormalTok{,}\FloatTok{0.01}\NormalTok{))}
\NormalTok{sp }\OtherTok{\textless{}{-}} \FunctionTok{list}\NormalTok{(}\AttributeTok{chain1=}\FunctionTok{c}\NormalTok{(}\FloatTok{0.01}\NormalTok{,}\FloatTok{0.99}\NormalTok{), }\AttributeTok{chain2=}\FunctionTok{c}\NormalTok{(}\FloatTok{0.99}\NormalTok{,}\FloatTok{0.01}\NormalTok{))}

\NormalTok{results }\OtherTok{\textless{}{-}} \FunctionTok{run.jags}\NormalTok{(}\StringTok{\textquotesingle{}basic\_hw.txt\textquotesingle{}}\NormalTok{, }\AttributeTok{n.chains=}\DecValTok{2}\NormalTok{)}
\end{Highlighting}
\end{Shaded}

\normalsize

Remember to check convergence and effective sample size!
\end{frame}

\begin{frame}
\begin{block}{Recap - psrf: potential scale reduction factor}
\protect\hypertarget{recap---psrf-potential-scale-reduction-factor}{}
\begin{itemize}
\tightlist
\item
  The psrf or Gelman-Rubin statistic is an estimated factor by which the
  scale of the current distribution for the target distribution might be
  reduced if the simulations were continued for an infinite number of
  iterations.
\item
  psrf estimates the potential decrease in the between-chains
  variability with respect to the within-chain variability.
\item
  If psrf is large, then longer simulation sequences are expected to
  either decrease between-chains variability or increase within-chain
  variability because the simulations have not yet explored the full
  posterior distribution.
\item
  If psrf \textless{} 1.1 for all model parameters, one can be fairly
  confident that convergence has been reached. Otherwise, longer chains
  or other means for improving the convergence may be needed.
\end{itemize}
\end{block}
\end{frame}

\begin{frame}
\begin{block}{Recap - SSeff: effective sample size}
\protect\hypertarget{recap---sseff-effective-sample-size}{}
\begin{itemize}
\item
  Effective sample size is a calculation of the equivalent number of
  independent samples that would contain the same posterior accuracy as
  the correlated samples from an MCMC. Thus, MCMC Efficiency is the
  number of effectively independent samples generated per second.
\item
  With high autocorrelation, the effective sample size will be lower.
\item
  We want SSeff \textgreater{} 1000
\end{itemize}
\end{block}
\end{frame}

\begin{frame}[fragile]
\scriptsize

\begin{Shaded}
\begin{Highlighting}[]
\FunctionTok{summary}\NormalTok{(results)}
\end{Highlighting}
\end{Shaded}

\normalsize

\scriptsize

\begin{longtable}[]{@{}lrrrrr@{}}
\toprule\noalign{}
& Lower95 & Median & Upper95 & SSeff & psrf \\
\midrule\noalign{}
\endhead
prev & 0.329 & 0.499 & 0.666 & 4250 & 2.285 \\
prob{[}1{]} & 0.256 & 0.344 & 0.438 & 13379 & 1.000 \\
prob{[}2{]} & 0.018 & 0.055 & 0.104 & 9529 & 1.000 \\
prob{[}3{]} & 0.071 & 0.133 & 0.200 & 13380 & 1.000 \\
prob{[}4{]} & 0.365 & 0.461 & 0.556 & 12986 & 1.000 \\
se{[}1{]} & 0.000 & 0.404 & 0.968 & 4347 & 13.481 \\
se{[}2{]} & 0.027 & 0.517 & 1.000 & 4653 & 15.311 \\
sp{[}1{]} & 0.037 & 0.595 & 1.000 & 4273 & 13.571 \\
sp{[}2{]} & 0.000 & 0.462 & 0.972 & 4538 & 15.171 \\
\bottomrule\noalign{}
\end{longtable}

\normalsize

\scriptsize

\begin{Shaded}
\begin{Highlighting}[]
\FunctionTok{plot}\NormalTok{(results)}
\end{Highlighting}
\end{Shaded}

\normalsize
\end{frame}

\begin{frame}
\scriptsize\includegraphics{day1_files/figure-beamer/unnamed-chunk-9-1.pdf}
\normalsize
\end{frame}

\begin{frame}
\scriptsize\includegraphics{day1_files/figure-beamer/unnamed-chunk-10-1.pdf}
\normalsize
\end{frame}

\begin{frame}
\scriptsize\includegraphics{day1_files/figure-beamer/unnamed-chunk-11-1.pdf}
\normalsize
\end{frame}

\begin{frame}
\begin{itemize}
\tightlist
\item
  Does anybody spot a problem?
\end{itemize}
\end{frame}

\begin{frame}[fragile]
\begin{block}{Label Switching}
\protect\hypertarget{label-switching}{}
How to interpret a test with Se=0\% and Sp=0\%?

\begin{itemize}
\tightlist
\item
  The test is perfect - we are just holding it upside down\ldots{}
\end{itemize}

We can force Se+Sp \textgreater= 1.

Jouden's Index Se + Sp -1 \textgreater= 0:

\scriptsize

\begin{Shaded}
\begin{Highlighting}[]
\NormalTok{  se[}\DecValTok{1}\NormalTok{] }\SpecialCharTok{\textasciitilde{}} \FunctionTok{dbeta}\NormalTok{(}\DecValTok{1}\NormalTok{, }\DecValTok{1}\NormalTok{)}
\NormalTok{  sp[}\DecValTok{1}\NormalTok{] }\SpecialCharTok{\textasciitilde{}} \FunctionTok{dbeta}\NormalTok{(}\DecValTok{1}\NormalTok{, }\DecValTok{1}\NormalTok{)}\FunctionTok{T}\NormalTok{(}\DecValTok{1}\SpecialCharTok{{-}}\NormalTok{se[}\DecValTok{1}\NormalTok{], )}
\end{Highlighting}
\end{Shaded}

\normalsize

Or:

\scriptsize

\begin{Shaded}
\begin{Highlighting}[]
\NormalTok{  se[}\DecValTok{1}\NormalTok{] }\SpecialCharTok{\textasciitilde{}} \FunctionTok{dbeta}\NormalTok{(}\DecValTok{1}\NormalTok{, }\DecValTok{1}\NormalTok{)}\FunctionTok{T}\NormalTok{(}\DecValTok{1}\SpecialCharTok{{-}}\NormalTok{sp[}\DecValTok{1}\NormalTok{], )}
\NormalTok{  sp[}\DecValTok{1}\NormalTok{] }\SpecialCharTok{\textasciitilde{}} \FunctionTok{dbeta}\NormalTok{(}\DecValTok{1}\NormalTok{, }\DecValTok{1}\NormalTok{)}
\end{Highlighting}
\end{Shaded}

\normalsize

This allows the test to be useless, but not worse than useless.
\end{block}
\end{frame}

\begin{frame}[fragile]
Alternatively we can have the weakly informative priors:

\scriptsize

\begin{Shaded}
\begin{Highlighting}[]
\NormalTok{  se[}\DecValTok{1}\NormalTok{] }\SpecialCharTok{\textasciitilde{}} \FunctionTok{dbeta}\NormalTok{(}\DecValTok{2}\NormalTok{, }\DecValTok{1}\NormalTok{)}
\NormalTok{  sp[}\DecValTok{1}\NormalTok{] }\SpecialCharTok{\textasciitilde{}} \FunctionTok{dbeta}\NormalTok{(}\DecValTok{2}\NormalTok{, }\DecValTok{1}\NormalTok{)}
\end{Highlighting}
\end{Shaded}

\normalsize

To give the model some information that we expect the test
characteristics to be closer to 100\% than 0\%.

\pause

Or we can use stronger priors for one or both tests.
\end{frame}

\begin{frame}[fragile]
\begin{block}{Priors}
\protect\hypertarget{priors}{}
A quick way to see the distribution of a prior:

\scriptsize

\begin{Shaded}
\begin{Highlighting}[]
\FunctionTok{curve}\NormalTok{(}\FunctionTok{dbeta}\NormalTok{(x, }\DecValTok{1}\NormalTok{, }\DecValTok{1}\NormalTok{), }\AttributeTok{from=}\DecValTok{0}\NormalTok{, }\AttributeTok{to=}\DecValTok{1}\NormalTok{)}
\end{Highlighting}
\end{Shaded}

\includegraphics{day1_files/figure-beamer/unnamed-chunk-15-1.pdf}

\begin{Shaded}
\begin{Highlighting}[]
\FunctionTok{qbeta}\NormalTok{(}\FunctionTok{c}\NormalTok{(}\FloatTok{0.025}\NormalTok{,}\FloatTok{0.975}\NormalTok{), }\AttributeTok{shape1=}\DecValTok{1}\NormalTok{, }\AttributeTok{shape2=}\DecValTok{1}\NormalTok{)}
\DocumentationTok{\#\# [1] 0.025 0.975}
\end{Highlighting}
\end{Shaded}

\normalsize
\end{block}
\end{frame}

\begin{frame}[fragile]
This was minimally informative, but how does that compare to a weakly
informative prior for e.g.~sensitivity?

\scriptsize

\begin{Shaded}
\begin{Highlighting}[]
\FunctionTok{curve}\NormalTok{(}\FunctionTok{dbeta}\NormalTok{(x, }\DecValTok{2}\NormalTok{, }\DecValTok{1}\NormalTok{), }\AttributeTok{from=}\DecValTok{0}\NormalTok{, }\AttributeTok{to=}\DecValTok{1}\NormalTok{)}
\end{Highlighting}
\end{Shaded}

\includegraphics{day1_files/figure-beamer/unnamed-chunk-16-1.pdf}

\begin{Shaded}
\begin{Highlighting}[]
\FunctionTok{qbeta}\NormalTok{(}\FunctionTok{c}\NormalTok{(}\FloatTok{0.025}\NormalTok{,}\FloatTok{0.975}\NormalTok{), }\AttributeTok{shape1=}\DecValTok{2}\NormalTok{, }\AttributeTok{shape2=}\DecValTok{1}\NormalTok{)}
\DocumentationTok{\#\# [1] 0.1581139 0.9874209}
\end{Highlighting}
\end{Shaded}

\normalsize
\end{frame}

\begin{frame}
\begin{block}{Choosing a prior}
\protect\hypertarget{choosing-a-prior}{}
What we want is e.g.~Beta(20,1)

But typically we have median and 95\% confidence intervals from a paper,
e.g.:

``The median (95\% CI) estimates of the sensitivity and specificity of
the shiny new test were 94\% (92-96\%) and 99\% (97-100\%)
respectively''

\begin{itemize}
\tightlist
\item
  How can we generate a Beta( , ) prior from this?
\end{itemize}
\end{block}
\end{frame}

\begin{frame}[fragile]
\begin{block}{The PriorGen package}
\protect\hypertarget{the-priorgen-package}{}
Median (95\% CI) estimates of Se and Sp were 94\% (92-96\%) and 99\%
(97-100\%)

\scriptsize

\begin{Shaded}
\begin{Highlighting}[]
\FunctionTok{library}\NormalTok{(}\StringTok{"PriorGen"}\NormalTok{)}
\DocumentationTok{\#\# Loading required package: rootSolve}
\DocumentationTok{\#\# Loading required package: nleqslv}
\FunctionTok{findbeta}\NormalTok{(}\AttributeTok{themedian =} \FloatTok{0.94}\NormalTok{, }\AttributeTok{percentile=}\FloatTok{0.95}\NormalTok{, }\AttributeTok{percentile.value =} \FloatTok{0.92}\NormalTok{)}
\DocumentationTok{\#\# $parameters}
\DocumentationTok{\#\#         a         b }
\DocumentationTok{\#\# 429.94997  27.75567 }
\DocumentationTok{\#\# }
\DocumentationTok{\#\# $summary}
\DocumentationTok{\#\#    Min. 1st Qu.  Median    Mean 3rd Qu.    Max. }
\DocumentationTok{\#\#  0.8920  0.9322  0.9400  0.9393  0.9471  0.9710 }
\DocumentationTok{\#\# }
\DocumentationTok{\#\# $input}
\DocumentationTok{\#\#        themedian       percentile percentile.value }
\DocumentationTok{\#\#             0.94             0.95             0.92 }
\DocumentationTok{\#\# }
\DocumentationTok{\#\# attr(,"class")}
\DocumentationTok{\#\# [1] "PriorGen"}
\end{Highlighting}
\end{Shaded}

\normalsize

Note: \texttt{themedian} could also be \texttt{themean}
\end{block}
\end{frame}

\begin{frame}[fragile]
\scriptsize

\begin{Shaded}
\begin{Highlighting}[]
\FunctionTok{curve}\NormalTok{(}\FunctionTok{dbeta}\NormalTok{(x, }\AttributeTok{shape1=}\FloatTok{429.95}\NormalTok{, }\AttributeTok{shape2=}\FloatTok{27.76}\NormalTok{))}
\end{Highlighting}
\end{Shaded}

\includegraphics{day1_files/figure-beamer/unnamed-chunk-18-1.pdf}
\normalsize
\end{frame}

\begin{frame}[fragile]
\begin{block}{Initial values}
\protect\hypertarget{initial-values}{}
Part of the problem before was also that we were specifying extreme
initial values:

\scriptsize

\begin{Shaded}
\begin{Highlighting}[]
\NormalTok{se }\OtherTok{\textless{}{-}} \FunctionTok{list}\NormalTok{(}\AttributeTok{chain1=}\FunctionTok{c}\NormalTok{(}\FloatTok{0.01}\NormalTok{,}\FloatTok{0.99}\NormalTok{), }\AttributeTok{chain2=}\FunctionTok{c}\NormalTok{(}\FloatTok{0.99}\NormalTok{,}\FloatTok{0.01}\NormalTok{))}
\NormalTok{sp }\OtherTok{\textless{}{-}} \FunctionTok{list}\NormalTok{(}\AttributeTok{chain1=}\FunctionTok{c}\NormalTok{(}\FloatTok{0.01}\NormalTok{,}\FloatTok{0.99}\NormalTok{), }\AttributeTok{chain2=}\FunctionTok{c}\NormalTok{(}\FloatTok{0.99}\NormalTok{,}\FloatTok{0.01}\NormalTok{))}
\end{Highlighting}
\end{Shaded}

\normalsize

\pause

Let's change these to:

\scriptsize

\begin{Shaded}
\begin{Highlighting}[]
\NormalTok{se }\OtherTok{\textless{}{-}} \FunctionTok{list}\NormalTok{(}\AttributeTok{chain1=}\FunctionTok{c}\NormalTok{(}\FloatTok{0.5}\NormalTok{,}\FloatTok{0.99}\NormalTok{), }\AttributeTok{chain2=}\FunctionTok{c}\NormalTok{(}\FloatTok{0.99}\NormalTok{,}\FloatTok{0.5}\NormalTok{))}
\NormalTok{sp }\OtherTok{\textless{}{-}} \FunctionTok{list}\NormalTok{(}\AttributeTok{chain1=}\FunctionTok{c}\NormalTok{(}\FloatTok{0.5}\NormalTok{,}\FloatTok{0.99}\NormalTok{), }\AttributeTok{chain2=}\FunctionTok{c}\NormalTok{(}\FloatTok{0.99}\NormalTok{,}\FloatTok{0.5}\NormalTok{))}
\end{Highlighting}
\end{Shaded}

\normalsize
\end{block}
\end{frame}

\begin{frame}[fragile]
\begin{block}{Exercise 1}
\protect\hypertarget{exercise-1}{}
Run the \texttt{hw\_definition} model under the following different
scenarios and interpret the results in each case.

\begin{enumerate}
\item
  Change the priors for \emph{Se} and \emph{Sp} and try Beta(2,1).
\item
  Estimate the Beta parameters for the \emph{Se} and \emph{Sp} of the
  shiny new test described.
\item
  Run the model using the beta priors for \emph{Se} and \emph{Sp} from
  Step 2.
\item
  Try to run the model with different initial values.
\item
  Force Se + Sp \textgreater{} 1. Be careful to specify initial values
  that are within the restricted parameter space.
\end{enumerate}
\end{block}
\end{frame}

\begin{frame}{Multi-population Hui-Walter models}
\protect\hypertarget{multi-population-hui-walter-models}{}
\end{frame}

\begin{frame}
\begin{block}{Hui-Walter models with multiple populations}
\protect\hypertarget{hui-walter-models-with-multiple-populations}{}
\begin{itemize}
\item
  Basically an extension of the single-population model
\item
  Works best with multiple populations each with differing prevalence

  \begin{itemize}
  \tightlist
  \item
    Including an unexposed population works well
  \item
    BUT be wary of assumptions regarding constant
    sensitivity/specificity across populations with very different types
    of infections
  \end{itemize}
\end{itemize}
\end{block}
\end{frame}

\begin{frame}
\begin{block}{Different prevalence in different populations (1st
assumption)}
\protect\hypertarget{different-prevalence-in-different-populations-1st-assumption}{}
What changes?

\begin{itemize}
\item
  In each population the data are summarized in a two\_x\_two table
  (2\^{}2 cells) again
\item
  or each population has a vector that contains all possible test
  results combinations
\end{itemize}
\end{block}
\end{frame}

\begin{frame}[fragile]
\begin{block}{Model specification}
\protect\hypertarget{model-specification}{}
\scriptsize

\begin{Shaded}
\begin{Highlighting}[]
\NormalTok{model\{}
  \ControlFlowTok{for}\NormalTok{(p }\ControlFlowTok{in} \DecValTok{1}\SpecialCharTok{:}\NormalTok{Populations)\{}
\NormalTok{    Tally[}\DecValTok{1}\SpecialCharTok{:}\DecValTok{4}\NormalTok{, p] }\SpecialCharTok{\textasciitilde{}} \FunctionTok{dmulti}\NormalTok{(prob[}\DecValTok{1}\SpecialCharTok{:}\DecValTok{4}\NormalTok{, p], TotalTests[p])}
    \CommentTok{\# Test1{-} Test2{-} Pop1}
\NormalTok{    prob[}\DecValTok{1}\NormalTok{, p] }\OtherTok{\textless{}{-}}\NormalTok{ (prev[p] }\SpecialCharTok{*}\NormalTok{ ((}\DecValTok{1}\SpecialCharTok{{-}}\NormalTok{se[}\DecValTok{1}\NormalTok{])}\SpecialCharTok{*}\NormalTok{(}\DecValTok{1}\SpecialCharTok{{-}}\NormalTok{se[}\DecValTok{2}\NormalTok{])) }\SpecialCharTok{+}\NormalTok{ ((}\DecValTok{1}\SpecialCharTok{{-}}\NormalTok{prev[p]) }\SpecialCharTok{*}\NormalTok{ (sp[}\DecValTok{1}\NormalTok{]}\SpecialCharTok{*}\NormalTok{sp[}\DecValTok{2}\NormalTok{]))}
    \DocumentationTok{\#\# snip \#\#}
      
\NormalTok{    prev[p] }\SpecialCharTok{\textasciitilde{}} \FunctionTok{dbeta}\NormalTok{(}\DecValTok{1}\NormalTok{, }\DecValTok{1}\NormalTok{)}
  \ErrorTok{\}}

\NormalTok{  se[}\DecValTok{1}\NormalTok{] }\SpecialCharTok{\textasciitilde{}} \FunctionTok{dbeta}\NormalTok{(se\_prior[}\DecValTok{1}\NormalTok{,}\DecValTok{1}\NormalTok{], se\_prior[}\DecValTok{1}\NormalTok{,}\DecValTok{2}\NormalTok{])}\FunctionTok{T}\NormalTok{(}\DecValTok{1}\SpecialCharTok{{-}}\NormalTok{sp[}\DecValTok{1}\NormalTok{], )}
\NormalTok{  sp[}\DecValTok{1}\NormalTok{] }\SpecialCharTok{\textasciitilde{}} \FunctionTok{dbeta}\NormalTok{(sp\_prior[}\DecValTok{1}\NormalTok{,}\DecValTok{1}\NormalTok{], sp\_prior[}\DecValTok{1}\NormalTok{,}\DecValTok{2}\NormalTok{])}
\NormalTok{  se[}\DecValTok{2}\NormalTok{] }\SpecialCharTok{\textasciitilde{}} \FunctionTok{dbeta}\NormalTok{(se\_prior[}\DecValTok{2}\NormalTok{,}\DecValTok{1}\NormalTok{], se\_prior[}\DecValTok{2}\NormalTok{,}\DecValTok{2}\NormalTok{])}\FunctionTok{T}\NormalTok{(}\DecValTok{1}\SpecialCharTok{{-}}\NormalTok{sp[}\DecValTok{2}\NormalTok{], )}
\NormalTok{  sp[}\DecValTok{2}\NormalTok{] }\SpecialCharTok{\textasciitilde{}} \FunctionTok{dbeta}\NormalTok{(sp\_prior[}\DecValTok{2}\NormalTok{,}\DecValTok{1}\NormalTok{], sp\_prior[}\DecValTok{2}\NormalTok{,}\DecValTok{2}\NormalTok{])}

  \CommentTok{\#data\# Tally, TotalTests, Populations, se\_prior, sp\_prior}
  \CommentTok{\#monitor\# prev, prob, se, sp}
  \CommentTok{\#inits\# prev, se, sp}
\ErrorTok{\}}
\end{Highlighting}
\end{Shaded}

\normalsize
\end{block}
\end{frame}

\begin{frame}
\begin{block}{Multiple populations: 2nd assumption}
\protect\hypertarget{multiple-populations-2nd-assumption}{}
\begin{itemize}
\tightlist
\item
  We typically assume that the sensitivity and specificity \emph{must}
  be consistent between populations

  \begin{itemize}
  \tightlist
  \item
    Do you have an endemic and epidemic population?
  \item
    Or vaccinated and unvaccinated?
  \item
    If so then the assumptions might not hold!
  \end{itemize}
\end{itemize}

\pause

\begin{itemize}
\tightlist
\item
  The populations can be artificial (e.g.~age groups) but must not be
  decided based on the diagnostic test results

  \begin{itemize}
  \tightlist
  \item
    It helps if the prevalence differs between the populations
  \end{itemize}
\end{itemize}
\end{block}
\end{frame}

\begin{frame}
\begin{block}{Multiple populations: special cases}
\protect\hypertarget{multiple-populations-special-cases}{}
\begin{itemize}
\tightlist
\item
  A small disease-free group is extremely helpful

  \begin{itemize}
  \tightlist
  \item
    Contains strong data regarding specificity
  \item
    As long as specificity can be assumed to be the same in the other
    populations
  \end{itemize}
\end{itemize}

\pause

\begin{itemize}
\tightlist
\item
  A small experimentally infected group MAY be helpful but it is often
  dangerous to assume that sensitivity is consistent!

  \begin{itemize}
  \tightlist
  \item
    \emph{Se} could differ based (I) age (II) epidemiological contex
    (III) stage of infection\ldots{}
  \end{itemize}
\end{itemize}
\end{block}
\end{frame}

\begin{frame}[fragile]
\begin{block}{Initial values}
\protect\hypertarget{initial-values-1}{}
We have to be careful to make sure that the length of initial values for
\texttt{prev} in each chain is equal to the number of populations

For example with 5 populations we need:

\scriptsize

\begin{Shaded}
\begin{Highlighting}[]
\NormalTok{prev }\OtherTok{\textless{}{-}} \FunctionTok{list}\NormalTok{(}\AttributeTok{chain1=}\FunctionTok{c}\NormalTok{(}\FloatTok{0.1}\NormalTok{, }\FloatTok{0.1}\NormalTok{, }\FloatTok{0.1}\NormalTok{, }\FloatTok{0.9}\NormalTok{, }\FloatTok{0.9}\NormalTok{), }\AttributeTok{chain2=}\FunctionTok{c}\NormalTok{(}\FloatTok{0.9}\NormalTok{, }\FloatTok{0.9}\NormalTok{, }\FloatTok{0.9}\NormalTok{, }\FloatTok{0.1}\NormalTok{, }\FloatTok{0.1}\NormalTok{))}
\end{Highlighting}
\end{Shaded}

\normalsize

The values you choose for different populations in the same chain can be
the same - just make sure you pick different values for the same
population between chains (i.e.~\emph{over-dispersed} initial values)

\emph{Note: specifying your own initial values is technically optional
with JAGS, but it is always a good idea (for now at least)!!!}
\end{block}
\end{frame}

\begin{frame}[fragile]
\begin{block}{Incorporating populations with known prevalence}
\protect\hypertarget{incorporating-populations-with-known-prevalence}{}
Up to now prevalence has been a parameter, but it can also be
(partially) observed:

\scriptsize

\begin{Shaded}
\begin{Highlighting}[]
\NormalTok{model\{}
  \ControlFlowTok{for}\NormalTok{(p }\ControlFlowTok{in} \DecValTok{1}\SpecialCharTok{:}\NormalTok{Populations)\{}
\NormalTok{    Tally[}\DecValTok{1}\SpecialCharTok{:}\DecValTok{4}\NormalTok{, p] }\SpecialCharTok{\textasciitilde{}} \FunctionTok{dmulti}\NormalTok{(prob[}\DecValTok{1}\SpecialCharTok{:}\DecValTok{4}\NormalTok{, p], TotalTests[p])}
    \CommentTok{\# Test1{-} Test2{-} Pop1}
\NormalTok{    prob[}\DecValTok{1}\NormalTok{, p] }\OtherTok{\textless{}{-}}\NormalTok{ (prev[p] }\SpecialCharTok{*}\NormalTok{ ((}\DecValTok{1}\SpecialCharTok{{-}}\NormalTok{se[}\DecValTok{1}\NormalTok{])}\SpecialCharTok{*}\NormalTok{(}\DecValTok{1}\SpecialCharTok{{-}}\NormalTok{se[}\DecValTok{2}\NormalTok{])) }\SpecialCharTok{+}\NormalTok{ ((}\DecValTok{1}\SpecialCharTok{{-}}\NormalTok{prev[p]) }\SpecialCharTok{*}\NormalTok{ (sp[}\DecValTok{1}\NormalTok{]}\SpecialCharTok{*}\NormalTok{sp[}\DecValTok{2}\NormalTok{]))}
    \DocumentationTok{\#\# snip \#\#}
      
\NormalTok{    prev[p] }\SpecialCharTok{\textasciitilde{}} \FunctionTok{dbeta}\NormalTok{(}\DecValTok{1}\NormalTok{, }\DecValTok{1}\NormalTok{)}
  \ErrorTok{\}}

  \DocumentationTok{\#\# snip \#\#}

  \CommentTok{\#data\# Tally, TotalTests, Populations, se\_prior, sp\_prior, prev}
  \CommentTok{\#monitor\# prev, prob, se, sp}
  \CommentTok{\#inits\# prev, se, sp}
\ErrorTok{\}}
\end{Highlighting}
\end{Shaded}

\normalsize
\end{block}
\end{frame}

\begin{frame}[fragile]
To fix the prevalence of population 1 we could do:

\scriptsize

\begin{Shaded}
\begin{Highlighting}[]
\NormalTok{Populations }\OtherTok{\textless{}{-}} \DecValTok{5}
\NormalTok{prev }\OtherTok{\textless{}{-}} \FunctionTok{rep}\NormalTok{(}\ConstantTok{NA}\NormalTok{, Populations)}
\NormalTok{prev[}\DecValTok{1}\NormalTok{] }\OtherTok{\textless{}{-}} \DecValTok{0}
\NormalTok{prev}
\DocumentationTok{\#\# [1]  0 NA NA NA NA}
\end{Highlighting}
\end{Shaded}

\normalsize

\pause

But you also need to account for this in the initial values:

\scriptsize

\begin{Shaded}
\begin{Highlighting}[]
\NormalTok{prev }\OtherTok{\textless{}{-}} \FunctionTok{list}\NormalTok{(}\AttributeTok{chain1=}\FunctionTok{c}\NormalTok{(}\ConstantTok{NA}\NormalTok{, }\FloatTok{0.1}\NormalTok{, }\FloatTok{0.1}\NormalTok{, }\FloatTok{0.9}\NormalTok{, }\FloatTok{0.9}\NormalTok{), }\AttributeTok{chain2=}\FunctionTok{c}\NormalTok{(}\ConstantTok{NA}\NormalTok{, }\FloatTok{0.9}\NormalTok{, }\FloatTok{0.9}\NormalTok{, }\FloatTok{0.1}\NormalTok{, }\FloatTok{0.1}\NormalTok{))}
\end{Highlighting}
\end{Shaded}

\normalsize

Note: we now have two definitions of \texttt{prev} in R!
\end{frame}

\begin{frame}[fragile]
\begin{block}{Data and initial value lists}
\protect\hypertarget{data-and-initial-value-lists}{}
There are actually multiple ways to specify data and initial values to
runjags, including via the \texttt{data} and \texttt{inits} arguments

We will use these to keep separate lists of data and initial values
(these could also be data frames, or environments)

\scriptsize

\begin{Shaded}
\begin{Highlighting}[]
\NormalTok{data }\OtherTok{\textless{}{-}} \FunctionTok{list}\NormalTok{(}
  \AttributeTok{Tally =}\NormalTok{ Tally,}
  \AttributeTok{TotalTests =} \FunctionTok{apply}\NormalTok{(Tally, }\DecValTok{2}\NormalTok{, sum),}
  \AttributeTok{Populations =} \FunctionTok{dim}\NormalTok{(Tally, }\DecValTok{2}\NormalTok{),}
  \AttributeTok{prev =} \FunctionTok{rep}\NormalTok{(}\ConstantTok{NA}\NormalTok{, Populations),}
  \AttributeTok{se\_prior =} \FunctionTok{matrix}\NormalTok{(}\DecValTok{1}\NormalTok{, }\AttributeTok{ncol=}\DecValTok{2}\NormalTok{, }\AttributeTok{nrow=}\DecValTok{2}\NormalTok{),}
  \AttributeTok{sp\_prior =} \FunctionTok{matrix}\NormalTok{(}\DecValTok{1}\NormalTok{, }\AttributeTok{ncol=}\DecValTok{2}\NormalTok{, }\AttributeTok{nrow=}\DecValTok{2}\NormalTok{)}
\NormalTok{)}
\NormalTok{data}\SpecialCharTok{$}\NormalTok{prev[}\DecValTok{1}\NormalTok{] }\OtherTok{\textless{}{-}} \DecValTok{0}
\end{Highlighting}
\end{Shaded}

\normalsize
\end{block}
\end{frame}

\begin{frame}[fragile]
\scriptsize

\begin{Shaded}
\begin{Highlighting}[]
\NormalTok{inits }\OtherTok{\textless{}{-}} \FunctionTok{list}\NormalTok{(}
  \AttributeTok{chain1 =} \FunctionTok{list}\NormalTok{(}
    \AttributeTok{prev =} \FunctionTok{c}\NormalTok{(}\ConstantTok{NA}\NormalTok{, }\FloatTok{0.1}\NormalTok{, }\FloatTok{0.1}\NormalTok{, }\FloatTok{0.9}\NormalTok{, }\FloatTok{0.9}\NormalTok{),}
    \AttributeTok{se =} \FunctionTok{c}\NormalTok{(}\FloatTok{0.5}\NormalTok{, }\FloatTok{0.99}\NormalTok{),}
    \AttributeTok{sp =} \FunctionTok{c}\NormalTok{(}\FloatTok{0.5}\NormalTok{, }\FloatTok{0.99}\NormalTok{)}
\NormalTok{  ),}
  \AttributeTok{chain2 =} \FunctionTok{list}\NormalTok{(}
    \AttributeTok{prev =} \FunctionTok{c}\NormalTok{(}\ConstantTok{NA}\NormalTok{, }\FloatTok{0.9}\NormalTok{, }\FloatTok{0.9}\NormalTok{, }\FloatTok{0.1}\NormalTok{, }\FloatTok{0.1}\NormalTok{),}
    \AttributeTok{se =} \FunctionTok{c}\NormalTok{(}\FloatTok{0.99}\NormalTok{, }\FloatTok{0.5}\NormalTok{),}
    \AttributeTok{sp =} \FunctionTok{c}\NormalTok{(}\FloatTok{0.99}\NormalTok{, }\FloatTok{0.5}\NormalTok{)}
\NormalTok{  )}
\NormalTok{)}

\NormalTok{results }\OtherTok{\textless{}{-}} \FunctionTok{run.jags}\NormalTok{(..., }\AttributeTok{data =}\NormalTok{ data, }\AttributeTok{inits =}\NormalTok{ inits)}
\end{Highlighting}
\end{Shaded}

\normalsize
\end{frame}

\begin{frame}[fragile]
See the help file for \texttt{?run.jags} for more details
\end{frame}

\begin{frame}[fragile]
Let's simulate a dataset to work on\ldots{}

\scriptsize

\begin{Shaded}
\begin{Highlighting}[]
\CommentTok{\# Set a random seed so that the data are reproducible:}
\FunctionTok{set.seed}\NormalTok{(}\DecValTok{2022{-}07{-}14}\NormalTok{)}

\NormalTok{sensitivity }\OtherTok{\textless{}{-}} \FunctionTok{c}\NormalTok{(}\FloatTok{0.9}\NormalTok{, }\FloatTok{0.6}\NormalTok{)}
\NormalTok{specificity }\OtherTok{\textless{}{-}} \FunctionTok{c}\NormalTok{(}\FloatTok{0.95}\NormalTok{, }\FloatTok{0.9}\NormalTok{)}
\NormalTok{N }\OtherTok{\textless{}{-}} \DecValTok{1000}

\CommentTok{\# Change the number of populations here:}
\NormalTok{Populations }\OtherTok{\textless{}{-}} \DecValTok{5}
\CommentTok{\# Change the variation in prevalence here:}
\NormalTok{(prevalence }\OtherTok{\textless{}{-}} \FunctionTok{runif}\NormalTok{(Populations, }\AttributeTok{min=}\FloatTok{0.1}\NormalTok{, }\AttributeTok{max=}\FloatTok{0.9}\NormalTok{))}

\NormalTok{data }\OtherTok{\textless{}{-}} \FunctionTok{tibble}\NormalTok{(}\AttributeTok{Population =} \FunctionTok{sample}\NormalTok{(}\FunctionTok{seq\_len}\NormalTok{(Populations), N, }\AttributeTok{replace=}\ConstantTok{TRUE}\NormalTok{)) }\SpecialCharTok{\%\textgreater{}\%}
  \FunctionTok{mutate}\NormalTok{(}\AttributeTok{Status =} \FunctionTok{rbinom}\NormalTok{(N, }\DecValTok{1}\NormalTok{, prevalence[Population])) }\SpecialCharTok{\%\textgreater{}\%}
  \FunctionTok{mutate}\NormalTok{(}\AttributeTok{Test1 =} \FunctionTok{rbinom}\NormalTok{(N, }\DecValTok{1}\NormalTok{, sensitivity[}\DecValTok{1}\NormalTok{]}\SpecialCharTok{*}\NormalTok{Status }\SpecialCharTok{+}\NormalTok{ (}\DecValTok{1}\SpecialCharTok{{-}}\NormalTok{specificity[}\DecValTok{1}\NormalTok{])}\SpecialCharTok{*}\NormalTok{(}\DecValTok{1}\SpecialCharTok{{-}}\NormalTok{Status))) }\SpecialCharTok{\%\textgreater{}\%}
  \FunctionTok{mutate}\NormalTok{(}\AttributeTok{Test2 =} \FunctionTok{rbinom}\NormalTok{(N, }\DecValTok{1}\NormalTok{, sensitivity[}\DecValTok{2}\NormalTok{]}\SpecialCharTok{*}\NormalTok{Status }\SpecialCharTok{+}\NormalTok{ (}\DecValTok{1}\SpecialCharTok{{-}}\NormalTok{specificity[}\DecValTok{2}\NormalTok{])}\SpecialCharTok{*}\NormalTok{(}\DecValTok{1}\SpecialCharTok{{-}}\NormalTok{Status)))}

\NormalTok{(twoXtwoXpop }\OtherTok{\textless{}{-}} \FunctionTok{with}\NormalTok{(data, }\FunctionTok{table}\NormalTok{(Test1, Test2, Population)))}
\NormalTok{(Tally }\OtherTok{\textless{}{-}} \FunctionTok{matrix}\NormalTok{(twoXtwoXpop, }\AttributeTok{ncol=}\NormalTok{Populations))}
\NormalTok{(TotalTests }\OtherTok{\textless{}{-}} \FunctionTok{apply}\NormalTok{(Tally, }\DecValTok{2}\NormalTok{, sum))}
\end{Highlighting}
\end{Shaded}

\normalsize
\end{frame}

\begin{frame}[fragile]
\scriptsize

\begin{Shaded}
\begin{Highlighting}[]
\NormalTok{Tally}
\DocumentationTok{\#\#      [,1] [,2] [,3] [,4] [,5]}
\DocumentationTok{\#\# [1,]   47   68  117  141  142}
\DocumentationTok{\#\# [2,]   48   47   24   34   22}
\DocumentationTok{\#\# [3,]   19   15   13   23   14}
\DocumentationTok{\#\# [4,]   76   68   38   30   14}
\end{Highlighting}
\end{Shaded}

\normalsize
\end{frame}

\begin{frame}
\begin{block}{Exercise 2}
\protect\hypertarget{exercise-2}{}
Start with 5 populations and analyse the data using the independent
prevalence model.

Now try with 2, 3, and 10 populations.

\begin{itemize}
\tightlist
\item
  How does this affect the confidence intervals for the diagnostic test
  parameters?
\end{itemize}

Now change the simulated prevalence so that it varies between 0.4-0.6
rather than 0.1-0.9.

\begin{itemize}
\tightlist
\item
  How does this affect the confidence intervals for the diagnostic test
  parameters?
\end{itemize}
\end{block}
\end{frame}

\begin{frame}[fragile]
\begin{block}{Solution 2}
\protect\hypertarget{solution-2}{}
This is what the model should look like:

\scriptsize

\begin{verbatim}

model{
  for(p in 1:Populations){
    Tally[1:4, p] ~ dmulti(prob[1:4, p], TotalTests[p])
  
    # Test1- Test2-
          prob[1,p] <- (prev[p] * ((1-se[1])*(1-se[2]))) + ((1-prev[p]) * ((sp[1])*(sp[2])))
    # Test1+ Test2-
    prob[2,p] <- (prev[p] * ((se[1])*(1-se[2]))) + ((1-prev[p]) * ((1-sp[1])*(sp[2])))
    # Test1- Test2+
    prob[3,p] <- (prev[p] * ((1-se[1])*(se[2]))) + ((1-prev[p]) * ((sp[1])*(1-sp[2])))
     # Test1+ Test2+
     prob[4,p] <- (prev[p] * ((se[1])*(se[2]))) + ((1-prev[p]) * ((1-sp[1])*(1-sp[2])))

    prev[p] ~ dbeta(1, 1)
  }
  se[1] ~ dbeta(se_prior[1,1], se_prior[1,2])T(1-sp[1], )
  sp[1] ~ dbeta(sp_prior[1,1], sp_prior[1,2])
  se[2] ~ dbeta(se_prior[2,1], se_prior[2,2])T(1-sp[2], )
  sp[2] ~ dbeta(sp_prior[2,1], sp_prior[2,2])

  #data# Tally, TotalTests, Populations, se_prior, sp_prior
  #monitor# prev, se, sp
  #inits# prev, se, sp
  #module# lecuyer
}
\end{verbatim}
\end{block}

\begin{block}{\normalsize}
\protect\hypertarget{section}{}
Here is the R code to run the model:

\scriptsize

\begin{Shaded}
\begin{Highlighting}[]
\FunctionTok{set.seed}\NormalTok{(}\DecValTok{2022{-}07{-}14}\NormalTok{)}

\CommentTok{\# Set up the se\_prior and sp\_prior variables (optional):}
\NormalTok{se\_prior }\OtherTok{\textless{}{-}} \FunctionTok{matrix}\NormalTok{(}\DecValTok{1}\NormalTok{, }\AttributeTok{nrow=}\DecValTok{2}\NormalTok{, }\AttributeTok{ncol=}\DecValTok{2}\NormalTok{)}
\NormalTok{sp\_prior }\OtherTok{\textless{}{-}} \FunctionTok{matrix}\NormalTok{(}\DecValTok{1}\NormalTok{, }\AttributeTok{nrow=}\DecValTok{2}\NormalTok{, }\AttributeTok{ncol=}\DecValTok{2}\NormalTok{)}

\CommentTok{\# Set up initial values for 5 populations:}
\NormalTok{se }\OtherTok{\textless{}{-}} \FunctionTok{list}\NormalTok{(}\AttributeTok{chain1=}\FunctionTok{c}\NormalTok{(}\FloatTok{0.5}\NormalTok{,}\FloatTok{0.99}\NormalTok{), }\AttributeTok{chain2=}\FunctionTok{c}\NormalTok{(}\FloatTok{0.99}\NormalTok{,}\FloatTok{0.5}\NormalTok{))}
\NormalTok{sp }\OtherTok{\textless{}{-}} \FunctionTok{list}\NormalTok{(}\AttributeTok{chain1=}\FunctionTok{c}\NormalTok{(}\FloatTok{0.5}\NormalTok{,}\FloatTok{0.99}\NormalTok{), }\AttributeTok{chain2=}\FunctionTok{c}\NormalTok{(}\FloatTok{0.99}\NormalTok{,}\FloatTok{0.5}\NormalTok{))}
\NormalTok{prev }\OtherTok{\textless{}{-}} \FunctionTok{list}\NormalTok{(}\AttributeTok{chain1=}\FunctionTok{c}\NormalTok{(}\FloatTok{0.1}\NormalTok{, }\FloatTok{0.1}\NormalTok{, }\FloatTok{0.1}\NormalTok{, }\FloatTok{0.9}\NormalTok{, }\FloatTok{0.9}\NormalTok{), }\AttributeTok{chain2=}\FunctionTok{c}\NormalTok{(}\FloatTok{0.9}\NormalTok{, }\FloatTok{0.9}\NormalTok{, }\FloatTok{0.9}\NormalTok{, }\FloatTok{0.1}\NormalTok{, }\FloatTok{0.1}\NormalTok{))}

\CommentTok{\# And run the model:}
\NormalTok{results\_5p }\OtherTok{\textless{}{-}} \FunctionTok{run.jags}\NormalTok{(}\StringTok{"multipopulation.txt"}\NormalTok{, }\AttributeTok{n.chains=}\DecValTok{2}\NormalTok{)}

\CommentTok{\# Remember to check convergence!}
\CommentTok{\# plot(results\_5p)}
\end{Highlighting}
\end{Shaded}

\normalsize
\end{block}
\end{frame}

\begin{frame}[fragile]
\scriptsize

\begin{Shaded}
\begin{Highlighting}[]
\FunctionTok{summary}\NormalTok{(results\_5p)}
\DocumentationTok{\#\#            Lower95    Median   Upper95      Mean         SD Mode        MCerr}
\DocumentationTok{\#\# prev[1] 0.64406865 0.7491024 0.8421899 0.7472234 0.05066793   NA 0.0008971740}
\DocumentationTok{\#\# prev[2] 0.53275306 0.6347588 0.7322373 0.6335632 0.05120282   NA 0.0008811056}
\DocumentationTok{\#\# prev[3] 0.22611620 0.3182248 0.4126892 0.3193326 0.04762100   NA 0.0007715615}
\DocumentationTok{\#\# prev[4] 0.16692047 0.2612688 0.3611840 0.2630157 0.05008596   NA 0.0009843755}
\DocumentationTok{\#\# prev[5] 0.06121917 0.1323756 0.2209423 0.1360347 0.04145460   NA 0.0007329051}
\DocumentationTok{\#\# se[1]   0.77078024 0.8459629 0.9219165 0.8462824 0.03914700   NA 0.0007558744}
\DocumentationTok{\#\# se[2]   0.55769314 0.6303388 0.7021480 0.6308721 0.03694907   NA 0.0006674124}
\DocumentationTok{\#\# sp[1]   0.85713598 0.9124706 0.9712733 0.9130531 0.02908931   NA 0.0005971085}
\DocumentationTok{\#\# sp[2]   0.87550725 0.9154121 0.9536456 0.9148896 0.02001620   NA 0.0003644522}
\DocumentationTok{\#\#         MC\%ofSD SSeff      AC.10      psrf}
\DocumentationTok{\#\# prev[1]     1.8  3189 0.08053215 1.0001076}
\DocumentationTok{\#\# prev[2]     1.7  3377 0.08337887 1.0000488}
\DocumentationTok{\#\# prev[3]     1.6  3809 0.07239327 0.9999959}
\DocumentationTok{\#\# prev[4]     2.0  2589 0.11490524 1.0001608}
\DocumentationTok{\#\# prev[5]     1.8  3199 0.09374986 1.0000866}
\DocumentationTok{\#\# se[1]       1.9  2682 0.10893491 1.0002422}
\DocumentationTok{\#\# se[2]       1.8  3065 0.08672611 1.0004764}
\DocumentationTok{\#\# sp[1]       2.1  2373 0.10590983 1.0003968}
\DocumentationTok{\#\# sp[2]       1.8  3016 0.09605686 1.0002598}
\end{Highlighting}
\end{Shaded}

\normalsize
\end{frame}

\begin{frame}[fragile]
To change the number of populations and range of prevalence you just
need to modify the simulation code, for example 3 populations with
prevalence between 0.4-0.6 can be obtained using:

\scriptsize

\begin{Shaded}
\begin{Highlighting}[]
\CommentTok{\# Change the number of populations here:}
\NormalTok{Populations }\OtherTok{\textless{}{-}} \DecValTok{3}
\CommentTok{\# Change the variation in prevalence here:}
\NormalTok{(prevalence }\OtherTok{\textless{}{-}} \FunctionTok{runif}\NormalTok{(Populations, }\AttributeTok{min=}\FloatTok{0.4}\NormalTok{, }\AttributeTok{max=}\FloatTok{0.6}\NormalTok{))}

\NormalTok{data }\OtherTok{\textless{}{-}} \FunctionTok{tibble}\NormalTok{(}\AttributeTok{Population =} \FunctionTok{sample}\NormalTok{(}\FunctionTok{seq\_len}\NormalTok{(Populations), N, }\AttributeTok{replace=}\ConstantTok{TRUE}\NormalTok{)) }\SpecialCharTok{\%\textgreater{}\%}
  \FunctionTok{mutate}\NormalTok{(}\AttributeTok{Status =} \FunctionTok{rbinom}\NormalTok{(N, }\DecValTok{1}\NormalTok{, prevalence[Population])) }\SpecialCharTok{\%\textgreater{}\%}
  \FunctionTok{mutate}\NormalTok{(}\AttributeTok{Test1 =} \FunctionTok{rbinom}\NormalTok{(N, }\DecValTok{1}\NormalTok{, sensitivity[}\DecValTok{1}\NormalTok{]}\SpecialCharTok{*}\NormalTok{Status }\SpecialCharTok{+}\NormalTok{ (}\DecValTok{1}\SpecialCharTok{{-}}\NormalTok{specificity[}\DecValTok{1}\NormalTok{])}\SpecialCharTok{*}\NormalTok{(}\DecValTok{1}\SpecialCharTok{{-}}\NormalTok{Status))) }\SpecialCharTok{\%\textgreater{}\%}
  \FunctionTok{mutate}\NormalTok{(}\AttributeTok{Test2 =} \FunctionTok{rbinom}\NormalTok{(N, }\DecValTok{1}\NormalTok{, sensitivity[}\DecValTok{2}\NormalTok{]}\SpecialCharTok{*}\NormalTok{Status }\SpecialCharTok{+}\NormalTok{ (}\DecValTok{1}\SpecialCharTok{{-}}\NormalTok{specificity[}\DecValTok{2}\NormalTok{])}\SpecialCharTok{*}\NormalTok{(}\DecValTok{1}\SpecialCharTok{{-}}\NormalTok{Status)))}

\NormalTok{(twoXtwoXpop }\OtherTok{\textless{}{-}} \FunctionTok{with}\NormalTok{(data, }\FunctionTok{table}\NormalTok{(Test1, Test2, Population)))}
\NormalTok{(Tally }\OtherTok{\textless{}{-}} \FunctionTok{matrix}\NormalTok{(twoXtwoXpop, }\AttributeTok{ncol=}\NormalTok{Populations))}
\NormalTok{(TotalTests }\OtherTok{\textless{}{-}} \FunctionTok{apply}\NormalTok{(Tally, }\DecValTok{2}\NormalTok{, sum))}
\CommentTok{\# Adjust initial values for 3 populations:}
\NormalTok{se }\OtherTok{\textless{}{-}} \FunctionTok{list}\NormalTok{(}\AttributeTok{chain1=}\FunctionTok{c}\NormalTok{(}\FloatTok{0.5}\NormalTok{,}\FloatTok{0.99}\NormalTok{), }\AttributeTok{chain2=}\FunctionTok{c}\NormalTok{(}\FloatTok{0.99}\NormalTok{,}\FloatTok{0.5}\NormalTok{))}
\NormalTok{sp }\OtherTok{\textless{}{-}} \FunctionTok{list}\NormalTok{(}\AttributeTok{chain1=}\FunctionTok{c}\NormalTok{(}\FloatTok{0.5}\NormalTok{,}\FloatTok{0.99}\NormalTok{), }\AttributeTok{chain2=}\FunctionTok{c}\NormalTok{(}\FloatTok{0.99}\NormalTok{,}\FloatTok{0.5}\NormalTok{))}
\NormalTok{prev }\OtherTok{\textless{}{-}} \FunctionTok{list}\NormalTok{(}\AttributeTok{chain1=}\FunctionTok{c}\NormalTok{(}\FloatTok{0.1}\NormalTok{, }\FloatTok{0.1}\NormalTok{, }\FloatTok{0.9}\NormalTok{), }\AttributeTok{chain2=}\FunctionTok{c}\NormalTok{(}\FloatTok{0.9}\NormalTok{, }\FloatTok{0.9}\NormalTok{, }\FloatTok{0.1}\NormalTok{))}
\CommentTok{\# And run the model:}
\NormalTok{results\_3p }\OtherTok{\textless{}{-}} \FunctionTok{run.jags}\NormalTok{(}\StringTok{"multipopulation.txt"}\NormalTok{, }\AttributeTok{n.chains=}\DecValTok{2}\NormalTok{)}
\CommentTok{\# Remember to check convergence!}
\CommentTok{\# plot(results\_3p)}
\CommentTok{\# summary(results\_3p)}
\end{Highlighting}
\end{Shaded}

\normalsize
\end{frame}

\begin{frame}[fragile]
Note that when the effective sample size is not enough - you either need
to run the model for longer in the first place, or extend it to get more
samples:

\scriptsize

\begin{Shaded}
\begin{Highlighting}[]
\CommentTok{\# Extend the model:}
\NormalTok{results\_3p }\OtherTok{\textless{}{-}} \FunctionTok{extend.jags}\NormalTok{(results\_3p, }\AttributeTok{sample=}\DecValTok{50000}\NormalTok{)}

\CommentTok{\# Remember to check convergence!}
\CommentTok{\# plot(results\_3p)}
\CommentTok{\# summary(results\_3p)}
\end{Highlighting}
\end{Shaded}

\normalsize
\end{frame}

\begin{frame}
As a general rule, the more populations you have, and the more the
prevalence varies between them, the better. However, this is conditional
on having a consistent sensitivity and specificity between your
populations!!!
\end{frame}

\begin{frame}
Time for another break
\end{frame}

\end{document}
